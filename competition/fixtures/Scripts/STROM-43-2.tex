\titulka{letného}{42.}

\begin{center}
\vskip -1.00cm
\textit{\textbf{Nezabudni si vytvoriť či aktualizovať profil na}  \url{https://seminar.strom.sk}.}
\end{center}

%\medskip
\znak{1}{Prvá}{18.~3.~2019}
\vskip -0.75cm

\uloha{1.}{Dokážte, že $\sqrt{n+1} + \sqrt{n-1}$ nie je racionálne číslo pre žiadne celočíselné $n$.}

\uloha{2.}{Nech $a$, $b$, $c$, $x$, $y$ a $z$ sú nezáporné reálne čísla také, že $a^2 + b^2 = c^2$ a $x^2+y^2=z^2$. Dokážte, že potom platí $(a+x)^2 + (b+y)^2 \leq (c+z)^2$ a zistite, kedy nastáva rovnosť.}

\uloha{3.}{Máme šachovnicu $n\times n$. Niektoré políčka (okrem ľavého horného a pravého dolného rohu) nafarbíme na červeno tak, že šachový kôň sa nevie dostať z ľavého horného rohu do pravého dolného rohu bez toho, aby musel stúpiť na červené políčko. Zistite, pre ktoré $n$ platí, že pri ľubovoľnom takomto ofarbení vieme nájsť tri po sebe idúce políčka na nejakej diagonále také, že aspoň dve z nich sú červené.}

\uloha{4.}{Lichobežník $ABCD$ je vpísaný do kružnice tak, že základňa lichobežníka $AB$ je jej priemer. Označme $E$ priesečník uhlopriečok lichobežníka $ABCD$, $S$ stred úsečky $AB$ a zostrojíme bod $X$ tak, aby bol $ASEX$ rovnobežník. Ukážte, že $|XA|=|XD|$.
}

\uloha{5.}{Dokážte, že ak funkcia $f:\mathbb{R} \rightarrow \mathbb{R}$ spĺňa nerovnosti $f(x)\leq x$ a $f(x+y) \leq f(x)+f(y)$ pre všetky $x$, $y$ reálne čísla, potom $f(x)=x$ pre všetky reálne $x$.
}

\uloha{6.}{V škole sa niektoré dvojice žiakov kamarátia a niektoré nie (kamarátstvo je obojstranné). Tímom nazývame skupinu práve 20 ľudí, v ktorej sa všetci navzájom kamarátia. Každý žiak je členom nejakého tímu, ale keď zrušíme ľubovoľné kamarátstvo, tak vždy bude existovať aspoň jeden žiak, ktorý nie je v žiadnom tíme. Tím, ktorý obsahuje žiaka, ktorý má kamarátov len v tomto tíme, nazveme \textit{\uv{tím so stredom}}. Dokážte, že pre ľubovoľnú dvojicu žiakov, ktorí sa kamarátia, existuje tím so stredom, ktorého sú obaja členmi.}


\smallskip

%\vfill
%\pagebreak
\znak{2}{Druhá}{24.~4.~2019}
\vskip -0.75cm

\uloha{1.}{V tabuľke $25\times 25$ sú čísla $+1$ a $-1$. Nech $a_i$ je súčin čísel v $i$-tom riadku a $b_j$ je súčin čísel v $j$-tom stĺpci. Dokážte, že súčet $a_1+b_1+\dots+a_{25}+b_{25}$ nie je rovný nule.
}

\uloha{2.}{Ukážte, že neexistuje aritmetická postupnosť s 3 členmi z nekonečnej geometrickej postupnosti ${\{2^k\}}^\infty_{k=0}$.
}

\uloha{3.}{V lichobežníku $ABCD$ sú $AB$ a $CD$ rovnobežné a navyše platí $|BC| = |AB|+|CD|$. Nech $F$ je stredom $AD$. Nájdite všetky možné veľkosti uhla $BFC$.
}

\uloha{4.}{Máme trojuholník $ABC$ a na strane $AB$ vyznačíme bod $S$ tak, aby $|AS|=|BS|$. Následne označme $I_1$ stred kružnice vpísanej trojuholníku $CAS$ a $I_2$ stred kružnice vpísanej trojuholníku $CBS$. Označme $k_1$ kružnicu opísanú trojuholníku $AI_1C$ a $k_2$ kružnicu opísanú trojuholníku $BI_2C$. Dokážte, že $k_1$ a $k_2$ sa okrem bodu $C$ pretínajú na priamke $CS$.
}

\uloha{5.}{Nájdite všetky trojice celých čísel $(a,\,b,\,c)$ také, že $3^a+3^b+3^c$ je druhou mocninou celého čísla.
}

\uloha{6.}{Nájdite všetky funkcie $f(x)$ na reálnych číslach spĺňajúce $f(t^2 + u) = t\cdot f(t) + f(u)$ pre všetky reálne čísla $t$ a $u$.}

\vfill
\textbf{Autori zadaní úloh:} Peter Kovács, Žaneta Semanišinová, Kristína Mišlanová, Roman Staňo, Daniel Onduš, Martin Masrna, Jakub Genči
%maju ostat rovnaky?