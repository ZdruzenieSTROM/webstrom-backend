\titulka{letného}{42.}

\begin{center}
\vskip -1.00cm
\textit{\textbf{Nezabudni si vytvoriť či aktualizovať profil na}  \url{https://seminar.strom.sk}.}
\end{center}

%\medskip
\znak{1}{Prvá}{19.~3.~2018}
\vskip -0.75cm

%Prešiel som to, može byť  Vodka
\uloha{1.}{Určte počet všetkých neusporiadaných trojíc dvojciferných prirodzených čísel $a,\ b,\ c$, ktorých súčin $abc$ má zápis, v~ktorom sú všetky cifry rovnaké.}

\uloha{2.}{Koľkými spôsobmi je možné ofarbiť čísla $1, 2, \dots , n$ červenou, zelenou a modrou farbou tak, že párne čísla nie sú zelené a žiadne dve susedné čísla nemajú rovnakú farbu?}

\uloha{3.}{Keďže Matúšova fenka Bodka je lenivá chodiť od jedného stromu k druhému, tak jej Matúš postavil systém minivláčikov. Medzi každými dvoma stromami v parku jazdí minivláčik práve v jednom smere. Bodka sa rozhodla, že si vyplní čas tým, že si vyberie nejaký prvý strom a potom sa prevezie minivláčikmi tak, že každý strom navštívi práve raz (a medzi stromami sa bude pohybovať len pomocou minivláčikov). Dokážte, že si tak Bodka vie vybrať bez ohľadu na to, ako Matúš minivláčiky postavil.}

\uloha{4.}{Nech $a,b,c\ge-1$ sú také reálne čísla, že $a^3+b^3+c^3=1$. Dokážte, že $a+b+c+a^2+b^2+c^2\le4$.}

\uloha{5.}{Daný je štvorec $ABCD$. Nájdite množinu všetkých bodov $P$ takých, že existuje rovnoramenný pravouhlý trojuholník $APQ$ s pravým uhlom  pri vrchole $P$ a bod $Q$ leží na strane $CD$.}

\uloha{6.}{V rovine je niekoľko priamok, pričom žiadne tri neprechádzajú jedným bodom a žiadne dve nie sú rovnobežné. Tieto priamky rozdelia rovinu na niekoľko oblastí. Dokážte, že môžeme dať do každej oblasti kladné číslo tak, aby pre každú priamku platilo, že súčet čísel v oblastiach na jednej strane priamky je rovnaký ako súčet čísel v oblastiach na druhej strane.}

\smallskip

%\vfill
%\pagebreak
\znak{2}{Druhá}{7.~5.~2018}
\vskip -0.75cm

\uloha{1.}{Nájdite všetky prirodzené čísla $n$ také, že všetky tri čísla $2n^{2}+1$, $3n^{2}+1$ a $6n^{2}+1$ sú druhými mocninami celých čísel.}

\uloha{2.}{Nech $p,q$ sú reálne čísla také, že rovnica $x^3+px+q=0$ má tri rôzne reálne riešenia. Dokážte, že potom platí $p\le0$.}

\uloha{3.}{Vodka a Tomáško hrajú hru na plániku $2018\times2$. Obaja majú $2\times1$ domino bloky, ktoré postupne po jednom pokladajú na plánik (v ťahoch sa striedajú a hráč musí v ťahu položiť blok na plán). Tomáško začína a môže pokladať bloky len v horizontálnom smere (ten, v ktorom má plánik rozmer 2018). Vodka môže pokladať bloky len vo vertikálnom smere. Hráč, ktorý nevie spraviť ťah, prehráva. Zistite, pre ktorého z hráčov existuje výherná stratégia.}

\uloha{4.}{Kružnica so stredom v bode $I$ je vpísaná do štvoruholníka $ABCD$. Polpriamky $BA$ a $CD$ sa pretínajú v bode $P$, polpriamky $AD$ a $BC$ sa pretínajú v bode $Q$. Za predpokladu, že $P$ leží na kružnici opísanej trojuholníku $AIC$ dokážte, že $Q$ tiež leží na tejto kružnici.}

\uloha{5.}{Nájdite najmenšie prvočíslo, ktoré sa nedá zapísať v tvare $|2^a-3^b|$, kde $a$, $b$ sú nezáporné celé čísla.}

\uloha{6.}{Nájdite všetky funkcie $f:\langle 0,+\infty) \rightarrow \langle 0,+\infty)$, ktoré vyhovujú súčasne nasledujúcim trom podmienkam:
\begin{enumerate}
\item pre ľubovoľné nezáporné reálne čísla $x$, $y$ také, že $x+y>0$, platí rovnosť $f(x\cdot f(y))\cdot f(y) = f(xy/(x+y))$,
\item $f(1) = 0$,
\item $f(x) > 0$ pre ľubovoľné $x>1$.
\end{enumerate}}

%\vfill
\textbf{Autori zadaní úloh:} Tomáš Babej, Žaneta Semanišinová, Kristína Mišlanová, Roman Staňo, Daniel Onduš, Martin Vodička, Jakub Genči
%maju ostat rovnaky?